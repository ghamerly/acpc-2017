\problemname{Optimizing Stock on Shelves}

Your friend, who is a store manager, is looking at optimising his sales
by reorganizing his products strategically on the shelves. Each product
being sold comes in one box, to be put on the shelves. The box contains
multiple such items, and we assume that we will not run out of items.
The boxes are of various sizes though. Note: the size here refers only
to the visible (front) length of the box, not how tall it is or how deep
it is. You cannot turn the box $90$ degrees to reduce or enlarge its
visible side.

The display area contains $4$ shelves, each of a fixed length (provided in
the input). When placing boxes on one shelf, you have to make sure that
they all fit (i.e., the sum of the boxes' lengths does not exceed the
size of the shelf). The input may contain more boxes or less than what
can fit in the entire display area. Each box contain a different
product.

The placement of a product on the shelves will have an impact on how
many will be sold. For example, being on the top ($4\textsuperscript{th}$)
shelf brings higher visibility usually, so more customers will see it
and buy it. Only the vertical placement is important here to the
expected sales of the product, not the horizontal placement (i.e., we
just want to know on which shelf, between $1$ and $4$, each box should go).
For each product, you can actually evaluate how much can be sold (in
dollars, referred to as ``expected sales'') when placed on each of the
shelves. For simplicity, this is a function of the form $f(i) = a + b * i$
where $i$ is the shelf number (between $1$ and $4$). Those functions are
different for each product, and are provided in the input.

What you have to do is to find the optimal placement of the boxes that
maximizes the sum of the expected sales of the products on display.

\section*{Input:}
The first line contains one positive integer, smaller than $20$,
representing the size (length) of the shelves in the display area. The
second line contains one positive integer $N$, smaller than $100$,
representing the number of boxes available, to be put on the shelves.
Then there are $N$ lines, one per box, each containing $3$ integers: the
first one (always positive, and smaller than $20$) represent the size
(length) of the box. The other two numbers are the $a$ and $b$
coefficients in the formula for the expected sales. Note that $a$ or
$b$ might be negative, but the function $f(i)$ will always evaluate to a
positive integer for the range $i=1\dots 4$.

\section*{Output:}
A single integer, representing the maximum total expected sales.

In the first example provided below, the optimal solution of $49$ is
achieved in the following way:

On row $4$, put boxes $(3, 1, 4)$ and $(2, 1, 2)$ ($\text{expected sales}=17+9=26$)

On row $3$, put boxes $(3, 2, 3)$ and $(1, -1, 2)$ ($\text{expected sales}=11+5=16$)

On row $2$, do not put any boxes

On row $1$, put box $(2, 8, -1)$ ($\text{expected sales}=7$)

In the second example, the optimal result of $91$ is achieved in the
following way:

On row $4$, put boxes $(2, 4, 4)$, $(1, 3, 3)$, and $(1, 1, 1)$ ($\text{expected sales}=20+15+5=40$)

On row $3$, put boxes $(3, 10, 4)$ and $(1, 1, 1)$ ($\text{expected sales}=22+4=26$)

On row $2$, put $4$ boxes $(1, 1, 1)$ ($\text{expected sales}=3+3+3+3=12$)

On row $1$, put $3$ boxes $(1, 1, 1)$ and the box $(1 9, -2)$ ($\text{expected sales}=2+2+2+7=13$)

