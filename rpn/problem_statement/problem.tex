\problemname{RPN Calculator}

Your goal is to write a postfix (``Reverse Polish Notation'')
calculator.  Unlike the probably more familiar ``infix'' notation, in
postfix notation the numbers are given first, then the desired
operation.  The following example is from Wikipedia.  The infix
notation
\begin{displaymath}
((15 / (7 - (1 + 1))) * 3) - (2 + (1 + 1))  
\end{displaymath}
can be written as
\begin{displaymath}
15\quad{}7\quad{}1\quad{}1\quad{}+\quad{}-\quad{}/\quad{}3\quad{}*\quad{}2\quad{}1\quad{}1\quad{}+\quad{}+\quad{}-  
\end{displaymath}
It can be evaluated by what
Wikipedia calls the ``left to right'' algorithm as follows:

\begin{verbatim}
15 7 1 1 + - / 3 * 2 1 1 + + - =
15 7     2 - / 3 * 2 1 1 + + - =
15         5 / 3 * 2 1 1 + + - =
             3 3 * 2 1 1 + + - =
                 9 2 1 1 + + - =
                 9 2     2 + - =
                 9         4 - =
                             5
\end{verbatim}


Your program will read and process lines of input one at a time, until
it reads \texttt{quit} on a line by itself. Each line of input will either be
an integer, or a string representing an operation. If the input is a
number, your program should save it on a stack.  If the input is an
arithmetic operation, your program should remove the two most recently
saved numbers, and replace them with the result of the operation.  If
before the operation, the most recently added number (i.e. the top of
the stack) is $b$, and the second most recently added number is $a$,
you should replace them with
\begin{displaymath}
  \begin{array}{cc}
    \mathrm{operation} & \mathrm{replacement}\\
    + & a+b\\
    - & a-b\\
    / & \lfloor a / b \rfloor\\
    {}* & a*b\\
    {}\verb|^| & a^b
  \end{array}
\end{displaymath}
The non-arithmetic operations should be implemented as follows:
\begin{center}
\begin{tabular}{ll}
dup & copy the top element of the stack \\
print& print the top element of the stack \\
pop& remove the top element of the stack\\
swap& interchange the top two elements on the stack \\
quit& Stop reading and processing input\\
\end{tabular}
\end{center}


\section*{Input}

Each line of input will either be an integer, or a string
$+, -, *, /, \verb|^|$, \texttt{dup}, \texttt{pop}, \texttt{print},
\texttt{quit}, or \texttt{swap}.
You may assume that there will always be sufficient arguments for each
operation (i.e. there is no ``stack underflow''), and that no division
by zero is attempted.

\section*{Output}
The output should be the output from the print operations in the input (if any).

