\problemname{Slider Puzzle}
\illustration{0.5}{slider-puzzle.png}{Image by {Scott Bateman}, Used under license}

A slider puzzle consists of a row of squares each of which contains a
single digit integer, like the illustration.

A circle marker starts on the initial square on the left and can be
moved to other squares along the row. At each step in the puzzle, you
may only move the marker the number of squares indicated by the integer
in the square it currently occupies, but it the marker can be moved 
either left or right along the row. The marker may not move past either 
end of the puzzle. For example, in the puzzle above the only legal 
first move is to move the marker three squares to the right because 
there is no room to move three spaces to the left.

The goal of the puzzle is to move the marker to the 0 at the far end of
the row, on the right. However, not every puzzle has a valid solution.

\section*{Input}
Input for this problem will be a number between $1$ and $100$ inclusive
representing the number of puzzles in the input, each puzzle is between
$10$ and $101$ digits long. Each individual puzzle will be on one line and
consist of a series of integers separated by spaces. 

\section*{Output}
Output should simply state whether ``Puzzle N is solvable.'' or 
``Puzzle N is not solvable.'' where $N$ refers to the puzzle instance.

