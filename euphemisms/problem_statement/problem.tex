\problemname{Euphemisms-Shmeuphemisms}
\illustration{0.25}{used_cars.jpg}{Image by \href{http://www.geograph.org.uk/photo/1925891}{Lewis Clarke
(geograph)}, Used under license}

%Political correctness is important in Bitville. A newly-appointed Editor-in-chief
%of the local Bitville newspaper is advised to use euphemisms, whenever possible.
As stated in Wikipedia, ``A euphemism is a generally innocuous word or expression used in place of one that may be found offensive or suggest something unpleasant''.
Thus, you are never {\it{lost}}, but ``geographically mislocated''.
%The military are among the most creative,
%with
Some other creative examples include ``penetrating deliverer of kinetic energy'' for bullet, and the now classic ``enhanced interrogation methods''.

The administration of the local Bitville newspaper, being risk-averse and mindful of political correctness, introduces a list of forbidden words,
along with the corresponding euphemisms to replace them with. Too late to teach an old dog new tricks! The staff members, who are supposed to replace
the forbidden words with the officially endorsed equivalents, concurred on the following strategy instead: they will continue writing as they did before, 
but will simply remove some characters from their articles, so that no forbidden word appears in the text. After all, a Bitville text
is merely a contiguous stretch of $0$s and $1$s, and as long as you remove the absolute minimum number of characters, you can hope no one will spot the difference!
Your job is, given a list of forbidden words $w_1,w_2,\ldots,w_n$ and the text $T$ to edit, to find the minimum number of characters to remove from $T$,
so that none of the strings $w_1,w_2,\ldots,w_n$ occurs as a substring in the text $T.$ 

\section*{Input}
The first line contains an integer~$ts, 1 \leq ts \leq 100$, denoting the number of test cases.
Then follow~$ts$ blocks, in the following format:
\begin{itemize}
  \item a single line with an integer~$n, 1\leq n \leq 30$ -- the number of forbidden Bitville-words
  \item $n$ lines, each line containing a forbidden word~$w_i,$ with length $1 \leq |w_i| \leq 13$ 
% \item an integer~$t_i, 1\leq t_i \leq 10\ 000$, which is the number of floppyseconds it takes for the read/write head of
% frequency~$i$ to move between the end points of its radial axis.
% \item an integer~$n_i, 1\leq n_i \leq 100$, denoting the number of intervals
% for which frequency~$i$ should play.
% \item $2n_i$ integers~$t_{i,j}, 0\leq t_{i,j} \leq 1\ 000\ 000$ where each $t_{i,j}$ denotes a toggle
% (on/off) for frequency~$i$ at time $t_{i,j}$. You can assume that these numbers are in strictly
% ascending order, i.e. $t_{i,1} < t_{i,2} < \dots < t_{i, 2n_i}$.
  \item finally, a line with the text $T$ to edit, with length $1 \leq |T| \leq 3000.$
\end{itemize}
Thus, a test case with $n$ forbidden words consists of $n+2$ lines. You are guaranteed that words and the text are non-empty, will not contain any leading or trailing blanks,
and are composed of solely $0$s and $1$s.
% All frequencies are initially switched off.

\section*{Output}
Per test case, a single line in the format ``{\texttt{Case \#t: ans}}'', where ``{\texttt{t}}'' stands for the number of the test case (starting from $1$), and ``{\texttt{ans}}'' is the answer to the problem, which is an integer indicating the minimum number of characters to remove for this test case.

